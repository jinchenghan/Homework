\documentclass[12pt]{article}%
\usepackage{amsfonts}
\usepackage{fancyhdr}
\usepackage{comment}
\usepackage[a4paper, top=2.5cm, bottom=2.5cm, left=2.2cm, right=2.2cm]%
{geometry}
\usepackage{times}
\usepackage{amsmath}
\usepackage{changepage}
\usepackage{amssymb}
\usepackage{graphicx}%
\setcounter{MaxMatrixCols}{30}
\newtheorem{theorem}{Theorem}
\newtheorem{acknowledgement}[theorem]{Acknowledgement}
\newtheorem{algorithm}[theorem]{Algorithm}
\newtheorem{axiom}{Axiom}
\newtheorem{case}[theorem]{Case}
\newtheorem{claim}[theorem]{Claim}
\newtheorem{conclusion}[theorem]{Conclusion}
\newtheorem{condition}[theorem]{Condition}
\newtheorem{conjecture}[theorem]{Conjecture}
\newtheorem{corollary}[theorem]{Corollary}
\newtheorem{criterion}[theorem]{Criterion}
\newtheorem{definition}[theorem]{Definition}
\newtheorem{example}[theorem]{Example}
\newtheorem{exercise}[theorem]{Exercise}
\newtheorem{lemma}[theorem]{Lemma}
\newtheorem{notation}[theorem]{Notation}
\newtheorem{problem}[theorem]{Problem}
\newtheorem{proposition}[theorem]{Proposition}
\newtheorem{remark}[theorem]{Remark}
\newtheorem{solution}[theorem]{Solution}
\newtheorem{summary}[theorem]{Summary}
\newenvironment{proof}[1][Proof]{\textbf{#1.} }{\ \rule{0.5em}{0.5em}}

\newcommand{\Q}{\mathbb{Q}}
\newcommand{\R}{\mathbb{R}}
\newcommand{\C}{\mathbb{C}}
\newcommand{\Z}{\mathbb{Z}}

\begin{document}

\title{Environmental Microbiology \\\
First Reading Assignment}
\author{Advisor: Zeev Ronen}
\date{Student: Jincheng Han\\\
ID: 850288812\\\
\today}
\maketitle
\section{What is the role of flagella in bio-film formation?
}
Flagella play a important role in bio-film formation. specifically, flagella involves the surface sensing and the initial stages of surface adhesion which lead the formation of a bio-film. 




\section{How C-di-GMP  affect motility or biofilm formation?
}
Under the high level of c-di-GMP, it's will promote the bio-films; under the low levels of c-di-GMP, it will promote motility. \\\
C-di-GMP affect the master regulator of flagenllar gene expression, FleQ. FleQ which inhibit the process of bio-film exopolysaccharide synthesis.

\section{What is the main structure in the flagella that is affected by sticking to surface?
}
The control of the swim-or-stick switch leading to biofilm formation involves the inhibition of flagellar synthesis and rotation coupled with increased synthesis of the polymers and structures that are required for long-term attachment to a surface and biofilm formation: that is, pili, fimbriae, holdfasts, capsules, and so on.



\section{Why flagellar motor stators is important for rotation?}     
A common link between these bacteria is a requirement for the proper function of the flagellar motor stators that channel ions into the cell to drive flagellar rotation. Proper motor rotation is critical in the initial step of biofilm development. The attachment of the cell body and flagellum to a surface stops the flagella motor. it inhibit flagellar motor function and response to a surface and its initiation of biofilm formation.


\section{What are the line of evidence that \textit{motB} gene is important in biofilm formation in  B. subtilis.  
}
First, deletion of \textit{motB} results in nonmotile cells.\\\
Second, disrupting proton flow through the MotAB
stator, and therefore flagellar rotation, increases DegU.P activity.\\\
Third, inhibition of flagellar rotation, either
by overexpressing EpsE or by binding flagella with anti-Hag (flagellin) antibody.
\\\
\\\
\\\
\\\
\\\
GitHub : https://github.com/jinchenghan
\end{document}